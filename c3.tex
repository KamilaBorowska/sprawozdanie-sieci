% -*- latex -*-
% spr.tex

% Wymagane pola:
% - Imie Nazwisko
% - grupa (dzin+godzina zaj.)
% - Nazwa i numer cwiczenia)
% - data

% Wymagane jest umieszczenie źródeł sprawozdania (plik *.tex, źródła rysunków)
% w ktalogu ~/labsk. Pliki tekstowe muszą byc w formacie pl_PL.UTF-8
% (nie wiem jak plik tekstowy może być w formacie pl_PL, to jest locale,
%  chyba chodziło po prostu o UTF-8?)

% Wymagany jest wydruk z obu stron kartki!
\documentclass[a4paper,11pt]{article}

\usepackage{polski}
\usepackage[utf8]{inputenc}     % niepotrzebne gdy pierwsza linia pliku to "%& --translate-file=il2-pl"

\hoffset=-3.0cm                 % Mniejszy lewy margines
\textwidth=18cm                 % szerzej
\evensidemargin=0pt

\voffset=-3cm                   % Mniejszy gorny margines
\textheight=27cm                % szerzej wzdluz

\setlength{\parindent}{0pt}             % No paragraph indentation
\setlength{\parskip}{\medskipamount}    % Space between paragraphs
\raggedbottom                           % bez rozciagania strony

\title{Sprawozdanie z sieci -- ćwiczenie 3\\Poniedziałek, 10:00}
\author{Konrad Borowski\\Paweł Szczepanowski}

\begin{document}
\thispagestyle{empty}                   % bez numeracji stron

\maketitle

\section{Wstęp}

Celem ćwiczenia było połączenie do sieci Wi-Fi używając systemów
operacyjnych Ubuntu i FreeBSD.

\section{Trasowanie}

Po połączeniu się do sieci Wi-Fi, ustawiane są odpowiednie trasy
przez urządzenie Wi-Fi. Dla przykładu dla sieci ZETiIS, wygląda
ona tak. Urządzenie \verb|wlp2s0| jest urządzeniem Wi-Fi.

\begin{verbatim}
k8% netstat -rn
Kernel IP routing table
Destination     Gateway         Genmask         Flags   MSS Window  irtt Iface
0.0.0.0         10.146.146.3    0.0.0.0         UG        0 0          0 eno2
0.0.0.0         10.146.146.3    0.0.0.0         UG        0 0          0 wlp2s0
10.146.0.0      0.0.0.0         255.255.0.0     U         0 0          0 eno2
10.146.0.0      0.0.0.0         255.255.0.0     U         0 0          0 wlp2s0
169.254.0.0     0.0.0.0         255.255.0.0     U         0 0          0 wlp2s0
\end{verbatim}

Dostęp do internetu jest udostępnianiy przez trasę której
\verb|Destination| to \verb|0.0.0.0|. Jest to trasa ogólna, używana
kiedy żadna inna nie jest dostępna. Sieć ZETiIS też ustawia trasy
\verb|10.146/16| i~\verb|169.254/16|, które umożliwiają dostęp do
lokalnych komputerów w~sieci ZETiIS.

Dla porównania, sieć pwwifi-students ma inną tablicę trasowania i~nie
ma dostępu do tych samych co volt.

\begin{verbatim}
k8% netstat -rn
Kernel IP routing table
Destination     Gateway         Genmask         Flags   MSS Window  irtt Iface
0.0.0.0         10.146.146.3    0.0.0.0         UG        0 0          0 eno2
0.0.0.0         10.68.0.1       0.0.0.0         UG        0 0          0 wlp2s0
1.1.1.1         10.68.0.1       255.255.255.255 UGH       0 0          0 wlp2s0
10.68.0.0       0.0.0.0         255.255.0.0     U         0 0          0 wlp2s0
10.146.0.0      0.0.0.0         255.255.0.0     U         0 0          0 eno2
169.254.0.0     0.0.0.0         255.255.0.0     U         0 0          0 wlp2s0
\end{verbatim}

Ciekawą właściwością sieci pwwifi-students jest istnienie trasy
\verb|1.1.1.1|. Dopóki użytkownik się nie zaloguje, nie jest możliwe
połączenie z~internetem (z~wyłączeniem serwera DNS), a~wszystkie
próby połączenia z serwerami na porcie 80 (co ciekawe tylko port 80)
wywołują przekierowanie na formularz logowania na tym adresie.

\section{Używanie połączenia Wi-Fi pomimo posiadania połączenia przewodowego}

Po połączeniu się z~siecią Wi-Fi, system preferuje połączenie przewodowe
(ponieważ jest szybsze, co jest prawdą, ale jednym z~ćwiczeń jest użycie
połączenia Wi-Fi, a~nie przewodowego).

\end{document}
