% -*- latex -*-
% spr.tex

% Wymagane pola:
% - Imie Nazwisko
% - grupa (dzin+godzina zaj.)
% - Nazwa i numer cwiczenia)
% - data

% Wymagane jest umieszczenie źródeł sprawozdania (plik *.tex, źródła rysunków)
% w ktalogu ~/labsk. Pliki tekstowe muszą byc w formacie pl_PL.UTF-8
% (nie wiem jak plik tekstowy może być w formacie pl_PL, to jest locale,
%  chyba chodziło po prostu o UTF-8?)

% Wymagany jest wydruk z obu stron kartki!
\documentclass[a4paper,11pt]{article}

\usepackage{polski}
\usepackage{hyperref}
\usepackage[utf8]{inputenc}     % niepotrzebne gdy pierwsza linia pliku to "%& --translate-file=il2-pl"

\hoffset=-3.0cm                 % Mniejszy lewy margines
\textwidth=18cm                 % szerzej
\evensidemargin=0pt

\voffset=-3cm                   % Mniejszy gorny margines
\textheight=27cm                % szerzej wzdluz

\setlength{\parindent}{0pt}             % No paragraph indentation
\setlength{\parskip}{\medskipamount}    % Space between paragraphs
\raggedbottom                           % bez rozciagania strony

\title{Sprawozdanie z sieci -- ćwiczenie 3\\Poniedziałek, 10:00}
\author{Konrad Borowski\\Paweł Szczepanowski}

\begin{document}
\thispagestyle{empty}                   % bez numeracji stron

\maketitle

\section{Wstęp}

Celem ćwiczenia było połączenie do sieci Wi-Fi używając systemów
operacyjnych Ubuntu i FreeBSD.

\section{Połączenie Wi-Fi na systemie Ubuntu}

System Ubuntu ma wbudowany NetworkManager, więc wykorzystano jego interfejs
tekstowy w~celu unikniecia konfliktów z~nim. Przeszukiwanie sieci jest
możliwe za pomocą polecenia \verb|nmcli d wifi list|.

\begin{verbatim}
szczepp3@k9 ~ % nmcli d wifi list
*  SSID                  MODE   CHAN  RATE       SIGNAL  BARS  SECURITY    
   lego                  Infra  6     54 Mbit/s  59      ***   --          
   konferencja           Infra  1     54 Mbit/s  64      ***   WEP         
   pwwifi-students       Infra  6     54 Mbit/s  39      **    --          
   konferencja           Infra  6     54 Mbit/s  35      **    WEP         
   pwwifi2               Infra  1     54 Mbit/s  50      **    WPA2 802.1X 
   pwwifi                Infra  11    54 Mbit/s  49      **    --          
   Huawei_msm8909_2500   Infra  6     54 Mbit/s  44      **    WPA2        
   pwwifi                Infra  1     54 Mbit/s  47      **    --          
   pwwifi2               Infra  11    54 Mbit/s  40      **    WPA2 802.1X 
   ZETiIS                Infra  6     54 Mbit/s  100     ****  WPA2 802.1X 
   TP-LINK_81E9CF        Infra  8     54 Mbit/s  52      **    WPA1 WPA2   
   konferencja           Infra  11    54 Mbit/s  49      **    WEP         
   pwwifi-students       Infra  11    54 Mbit/s  25      *     --          
   konferencja           Infra  11    54 Mbit/s  39      **    WEP         
   pwwifi-students       Infra  1     54 Mbit/s  50      **    --          
   pwwifi                Infra  11    54 Mbit/s  37      **    --          
   Airbox-E436           Infra  11    54 Mbit/s  67      ***   WPA2        
   pwwifi2               Infra  11    54 Mbit/s  42      **    WPA2 802.1X 
   Sieć Wi-Fi (WE-Lech)  Infra  1     54 Mbit/s  54      **    WPA2        
   pwwifi2               Infra  1     54 Mbit/s  47      **    WPA2 802.1X 
   pwwifi2               Infra  11    54 Mbit/s  44      **    WPA2 802.1X 
   SKNET                 Infra  1     54 Mbit/s  37      **    WPA2        
   --                    Infra  4     54 Mbit/s  29      *     WPA2        
   pwwifi2               Infra  6     54 Mbit/s  29      *     WPA2 802.1X 
   konferencja           Infra  1     54 Mbit/s  35      **    WEP         
   konferencja           Infra  11    54 Mbit/s  35      **    WEP         
   pwwifi                Infra  6     54 Mbit/s  42      **    --
\end{verbatim}

Możliwe jest połączenie do której z~dostępnych sieci za pomocą polecenia
\verb|nmcli d wifi connect|. Nie jest konieczne korzystanie z~polecenia
\verb|sudo|, wtedy wyświetla się komunikat w~interfejsie graficznym
pytający o~hasło.

\begin{verbatim}
szczepp3@k9 ~ % sudo nmcli d wifi connect pwwifi-students
Device 'wlp2s0' successfully activated with 'df271965-dd70-42c2-9704-04dd8104c250'.
\end{verbatim}

Aktywowanie już zapisanego połączenia jest też możliwe, używając polecenia
\verb|nmcli c up|.

\begin{verbatim}
szczepp3@k9 ~ % sudo nmcli c up ZETiIS
Connection successfully activated (D-Bus active path: /org/freedesktop/NetworkManager/ActiveConnection/3)
\end{verbatim}

\section{Połączenie Wi-Fi na systemie FreeBSD}

Niestety ze względu na problemy z~trasownikiem nie udało się przeprowadzić
tej procedury. Udało się połączyć z~trasownikiem i~włączyć sieć Wi-Fi
(dzięki wykorzystaniu interfejsu sieciowego, znalezionego za pomocą
polecenia \verb|nmap|), ale trasownik nie miał
zainstalowanych odpowiednich sterowników. Nie mniej jednak, ogólny opis,
na podstawie dokumentacji systemu FreeBSD wygląda tak.

Modyfikując \verb|/etc/wpa_supplicant.conf| można wprowadzić parametry
sieci z~którą chcemy się połączyć. \verb|myssid| w~tym wypadku to nazwa sieci,
a~\verb|psk| to hasło do sieci.

\begin{verbatim}
network={
    ssid="myssid"
    psk="wpa2 psk"
}
\end{verbatim}

Następnie można zmodyfikować plik konfiguracyjny systemu aby włączył sieć.
\verb|ath0| to nazwa urządzenia sieciowego. Możliwe jest uzyskanie listy
urządzeń za pomocą polecenia \verb|sysctl net.wlan.devices|
(\verb/ifconfig | grep -B3 -i wireless/ w~starszych wersjach).

\begin{verbatim}
wlans_ath0="wlan0"
ifconfig_wlan0="WPA SYNCDHCP"
\end{verbatim}

Aby aktywować te ustawienia bez restartowania komputera, można wykorzystać
polecenie \verb|service netif restart|.

\section{Trasowanie}

Po połączeniu się do sieci Wi-Fi, ustawiane są odpowiednie trasy
przez urządzenie Wi-Fi. Dla przykładu dla sieci ZETiIS, wygląda
ona tak. Urządzenie \verb|wlp2s0| jest urządzeniem Wi-Fi.

\begin{verbatim}
k8% netstat -rn
Kernel IP routing table
Destination     Gateway         Genmask         Flags   MSS Window  irtt Iface
0.0.0.0         10.146.146.3    0.0.0.0         UG        0 0          0 eno2
0.0.0.0         10.146.146.3    0.0.0.0         UG        0 0          0 wlp2s0
10.146.0.0      0.0.0.0         255.255.0.0     U         0 0          0 eno2
10.146.0.0      0.0.0.0         255.255.0.0     U         0 0          0 wlp2s0
169.254.0.0     0.0.0.0         255.255.0.0     U         0 0          0 wlp2s0
\end{verbatim}

Dostęp do internetu jest udostępnianiy przez trasę której
\verb|Destination| to \verb|0.0.0.0|. Jest to trasa ogólna, używana
kiedy żadna inna nie jest dostępna. Sieć ZETiIS też ustawia trasy
\verb|10.146/16| i~\verb|169.254/16|, które umożliwiają dostęp do
lokalnych komputerów w~sieci ZETiIS.

Dla porównania, sieć pwwifi-students ma inną tablicę trasowania i~nie
ma dostępu do tych samych co volt.

\begin{verbatim}
k8% netstat -rn
Kernel IP routing table
Destination     Gateway         Genmask         Flags   MSS Window  irtt Iface
0.0.0.0         10.146.146.3    0.0.0.0         UG        0 0          0 eno2
0.0.0.0         10.68.0.1       0.0.0.0         UG        0 0          0 wlp2s0
1.1.1.1         10.68.0.1       255.255.255.255 UGH       0 0          0 wlp2s0
10.68.0.0       0.0.0.0         255.255.0.0     U         0 0          0 wlp2s0
10.146.0.0      0.0.0.0         255.255.0.0     U         0 0          0 eno2
169.254.0.0     0.0.0.0         255.255.0.0     U         0 0          0 wlp2s0
\end{verbatim}

Ciekawą właściwością sieci pwwifi-students jest istnienie trasy
\verb|1.1.1.1|. Dopóki użytkownik się nie zaloguje, nie jest możliwe
połączenie z~internetem (z~wyłączeniem serwera DNS), a~wszystkie
próby połączenia z serwerami na porcie 80 (co ciekawe tylko port 80)
wywołują przekierowanie na formularz logowania na tym adresie.

\section{Używanie połączenia Wi-Fi pomimo posiadania połączenia przewodowego}

Po połączeniu się z~siecią Wi-Fi, system preferuje połączenie przewodowe
(ponieważ jest szybsze, co jest prawdą, ale jednym z~ćwiczeń jest użycie
połączenia Wi-Fi, a~nie przewodowego). Można sprawić aby system preferował
użycie połączenia bezprezewodowego przez zwiększenie parametru \verb|metric|
połączenia przewodowego.

\begin{verbatim}
k8% route
Kernel IP routing table
Destination     Gateway         Genmask         Flags Metric Ref    Use Iface
default         vol.iem.pw.edu. 0.0.0.0         UG    0      0        0 eno2
default         10.68.0.1       0.0.0.0         UG    600    0        0 wlp2s0
1.1.1.1         10.68.0.1       255.255.255.255 UGH   600    0        0 wlp2s0
10.68.0.0       *               255.255.0.0     U     600    0        0 wlp2s0
10.146.0.0      *               255.255.0.0     U     0      0        0 eno2
link-local      *               255.255.0.0     U     1000   0        0 wlp2s0
k8% sudo route del default
k8% sudo route add default gw vol.iem.pw.edu.pl metric 1024 dev eno2
k8% route
Kernel IP routing table
Destination     Gateway         Genmask         Flags Metric Ref    Use Iface
default         10.68.0.1       0.0.0.0         UG    600    0        0 wlp2s0
default         vol.iem.pw.edu. 0.0.0.0         UG    1024   0        0 eno2
1.1.1.1         10.68.0.1       255.255.255.255 UGH   600    0        0 wlp2s0
10.68.0.0       *               255.255.0.0     U     600    0        0 wlp2s0
10.146.0.0      *               255.255.0.0     U     0      0        0 eno2
link-local      *               255.255.0.0     U     1000   0        0 wlp2s0
\end{verbatim}

Można sprawdzić czy to działa przez użycie polecenia \verb|traceroute|.

\begin{verbatim}
k8% traceroute 8.8.8.8
traceroute to 8.8.8.8 (8.8.8.8), 64 hops max
  1   10.68.0.1  40,031ms  12,796ms  12,284ms
  2   194.29.130.1  9,849ms  3,052ms  4,128ms
  3   194.29.132.164  3,447ms  9,803ms  12,595ms
  4   148.81.253.69  9,619ms  7,026ms  19,742ms
  5   212.191.224.73  8,026ms  9,459ms  8,575ms
  6   72.14.203.178  19,367ms  10,345ms  15,127ms
  7   66.249.95.15  19,892ms  9,572ms  235,889ms
  8   216.239.57.240  41,992ms  27,009ms  27,207ms
  9   66.249.95.226  48,669ms  *  30,558ms
 10   209.85.246.119  30,546ms  33,093ms  50,411ms
 11   *  *  *
 12   8.8.8.8  34,838ms  30,276ms  30,617ms
\end{verbatim}

Jak widać, połączenie przechodzi przez bramę domyślną (pod adresem \verb|10.68.0.1|)
i~następują opóźnienia związane z~korzystaniem połączenia Wi-Fi. Na dodatek,
próba korzystania z~przeglądarki internetowej przekierowuje na stronę logowania
jeżeli użytkownik nie jest zalogowany.

Mimo przeprowadzenia tej zmiany, nadal jest możliwość korzystania z~sieci wydzielonych
na połączeniu przewodowym (\verb|10.146/16|).

\section{Sieć peer-to-peer}

Jest możliwe korzystanie z~interfejsu Wi-Fi w~trybie ad hoc. Ten tryb jest rzadko
stosowany, jednakże jest dostępny. Możliwe jest korzystanie z~tego trybu przez
wykorzystanie następujących poleceń. \verb|MySSID| jest nazwą sieci, a \verb|2422|
jest częstotliwością (w~MHz) na której sieć jest uruchamiana. Lista dopasowań
częstotliwości do kanałów jest dostępna na stronie
\url{https://en.wikipedia.org/wiki/List_of_WLAN_channels}.

\begin{verbatim}
szczepp3@k9 ~ % sudo iw wlp2s0 set type ibss
szczepp3@k9 ~ % sudo ip link set wlp2s0 up
szczepp3@k9 ~ % sudo iw wlp2s0 ibss join MySSID 2422
\end{verbatim}

\end{document}
