% -*- latex -*-
% spr.tex

% Wymagane pola:
% - Imie Nazwisko
% - grupa (dzin+godzina zaj.)
% - Nazwa i numer cwiczenia)
% - data

% Wymagane jest umieszczenie źródeł sprawozdania (plik *.tex, źródła rysunków)
% w ktalogu ~/labsk. Pliki tekstowe muszą byc w formacie pl_PL.UTF-8
% (nie wiem jak plik tekstowy może być w formacie pl_PL, to jest locale,
%  chyba chodziło po prostu o UTF-8?)

% Wymagany jest wydruk z obu stron kartki!
\documentclass[a4paper,11pt]{article}

\usepackage{polski}
\usepackage{hyperref}
\usepackage[utf8]{inputenc}     % niepotrzebne gdy pierwsza linia pliku to "%& --translate-file=il2-pl"

\hoffset=-3.0cm                 % Mniejszy lewy margines
\textwidth=18cm                 % szerzej
\evensidemargin=0pt

\voffset=-3cm                   % Mniejszy gorny margines
\textheight=27cm                % szerzej wzdluz

\setlength{\parindent}{0pt}             % No paragraph indentation
\setlength{\parskip}{\medskipamount}    % Space between paragraphs
\raggedbottom                           % bez rozciagania strony

\title{Sprawozdanie z sieci -- ćwiczenie 5\\Poniedziałek, 10:00}
\author{Konrad Borowski\\Paweł Szczepanowski}

\begin{document}
\thispagestyle{empty}                   % bez numeracji stron

\maketitle
\tableofcontent
\section{Wstęp}

Celem zadania było podłączenie serwisów plikowych i~stworzenie serwerów
plików SMB, NFS i~iSCSI.

NFS (Network File System) to protokół zdalnego udostępniania systemu plików. Port, który standardowo przydzielony jest do tej usługi to 2049.

\section{Serwer NFS w~FreeBSD}

Został zmodyfikowany plik \verb|/etc/exports|.

\begin{verbatim}
/ -ro 192.168.10.1
\end{verbatim}

Ta linia pozwala na dostęp do folderu root dla komputera 192.168.10.1.

Następnie można uruchomić serwer.

\begin{verbatim}
root@:~ # service nfsd onestart 
NFSv4 is disabled
Starting rpcbind.
Starting mountd.
Starting nfsd.
\end{verbatim}

Można następnie używać tego katalogu na tym komputerze.

\begin{verbatim}
root@:~ # service nfsclient onestart
NFS access cache time=60            
root@:~ # mount 192.168.10.2:/ /mnt 
\end{verbatim}

Oraz zobaczyć jego zawartość.

\begin{verbatim}
root@:/mnt # ls               
.11.1-PRERELEASE        mnt   
.cshrc                  net   
.profile                proc  
COPYRIGHT               pub   
bin                     rescue
boot                    root  
dev                     sbin  
etc                     sys   
lib                     tmp   
libexec                 usr   
media                   var   
\end{verbatim}

\section{Podłączanie zasobów NFS w~systemie Arch Linux}

Najpierw trzeba zainstalować program obsługujący system plików NFS.

\begin{verbatim}
pacman -S nfs-utils
\end{verbatim}

Można podłączyć zasób sieciowy za pomocą polecenia \verb|mount|.
Lokalizacja folderu sieciowego jest podzielona na dwie części,
pierwsza to nazwa lokalizacji sieciowej, a druga to ścieżka.
Opcja \verb|nolock| wyłącza mechanizm blokowania plików który jest
problematyczny w~warunkach sieciowych.

\begin{verbatim}
root@arch-nfs / # mount -o nolock vol:/ /mnt

vol:/ on /mnt type nfs (rw,relatime,vers=3,rsize=65536,
wsize=65536,namlen=255,hard,nolock,proto=tcp,timeo=600,
retrans=2,sec=sys,mountaddr=10.146.146.3,mountvers=3,
mountport=790,mountproto=udp,local_lock=all,addr=10.146.146.3)
\end{verbatim}

Następnie można zobaczyć listę zamontowanych folderów
używając polecenia \verb|df -h|.

\begin{verbatim}
root@arch-nfs / # df -h
System plików            rozm. użyte dost. %uż. zamont. na
[...]
vol:/                     125G  114G  1,3G  99% /mnt
\end{verbatim}

\section{Serwer NFS w~systemie Arch Linux}

Lista plików eksportowanych jest przechowywana w~pliku \verb|/etc/exports|.
Dla przykładu jeżeli chcemy mieć folder \verb|/srv/nfs| dostępny dla lokalnej
sieci 10.0.0.0 można zapisać następującą treść w~pliku \verb|/etc/exports|.

\begin{verbatim}
/srv/nfs       10.0.0.0/8(rw,fsid=root,no_subtree_check)
\end{verbatim}

A następnie uruchomić serwer.

\begin{verbatim}
sudo service nfs-server start
\end{verbatim}

Można sprawdzić widoczność używając następującego polecenia.

\begin{verbatim}
showmount -e localhost
\end{verbatim}


\section{Eksportowanie w środowisku FreeBSD}





\subsection{Uruchomienie serwera NFS}
Uruchomienie serwera polega na dodaniu do pliku \texttt{/etc/rc.conf} linii:

\footnotesize\begin{verbatim}
# Portmaper dla wiekszosci serwisow sieciowych (nis, nfs)
rpcbind_enable=YES		       
# NFS-serwer
nfs_server_enable=YES	    
nfsv4_server_enable=YES        
mountd_enable=YES		        
mountd_flags="-rl"		       
\end{verbatim}\normalsize

Teraz należy ustalić, które foldery mają zostać wyeksportowane. Konfigurujemy to w pliku \texttt{/etc/exports}. Dla naszego folderu należy dopisać:
\footnotesize\begin{verbatim}
#    scieżka        parametry    dozwolone maszyny
/tmp/cw4nfs     -alldirs     localhost
\end{verbatim}\normalsize
Parametr '-alldirs' pozwala na montowanie podkatalogów. Inną użyteczną opcją jest '-ro', która ogranicza prawa użytkownika korzystającego z zasobów zdalnych do odczytu.


Aby uruchomić serwer NFS należy wykonać jedno z poniższych:
\begin{itemize}
	\item \texttt{\# service nfsd start}
	\item \texttt{\# nfsd -u -t -n 4}
\end{itemize}


Użyto polecenia:
\footnotesize\begin{verbatim}
% # service nfsd start
Starting nfsuserd.
Starting nfsd.
\end{verbatim}\normalsize

Po uruchomieniu usługi nfsd, należy uruchomić rpcbind oraz mountd.
\footnotesize\begin{verbatim}
% # service mountd start
Starting mountd.
\end{verbatim}\normalsize



Po zmianie w pliku /etc/exports należy przeładować ustawienia poleceniem:\\ \texttt{\# service mountd onereload}.

Za pomocą polecenia mount zweryfikowano, że folder został został prawidłowo wyeksportowany przez NFS:

\footnotesize\begin{verbatim}
% mount
/dev/md6 on /tmp/cw4nfs (ufs, NFS exported, local)
\end{verbatim}\normalsize


\subsection{Uruchomienie klienta NFS}
W celu konfiguracji klienta NFS wystarczy dopisać do pliku \texttt{/etc/rc.conf}:
\footnotesize\begin{verbatim}
nfs_client_enable=YES		# This host is an NFS client (or NO).
\end{verbatim}\normalsize

\subsection{Zamontowanie katalogu zdalnego}

Usługi udostępniane przez komputer zdalny można pozyskać poleceniem \texttt{showmount}.
\footnotesize\begin{verbatim}
% # showmount -e localhost
Exports list on localhost:
/tmp/cw4nfs                    localhost 

% # showmount -e volt
 % showmount -e volt
 Exports list on volt:
 /tmp/obj                           Everyone
 /home/stud                         ldap fed amp fre ubu len vxjac 
 /                                  Everyone
 /nfs                               lap 
\end{verbatim}\normalsize


Można utworzyć katalog i podpiąć zasób zdalny do utworzonego katalogu.

\footnotesize\begin{verbatim}
% mkdir /tmp/cw4nfs_client
% # mount localhost:/tmp/cw4nfs /tmp/cw4nfs_client
% mount
/dev/md6 on /tmp/cw4nfs (ufs, NFS exported, local)
localhost:/tmp/cw4nfs on /tmp/cw4nfs_client (nfs)
\end{verbatim}\normalsize

Aby możliwe było montowanie zdalne, a nie w obrębie jednego komputera, należy podać inny adres niż localhost.




\section{Serwer SMB}

Można zainstalować go korzystając z~polecenia.

\begin{verbatim}
sudo pacman -S samba
\end{verbatim}

Jego domyślna konfiguracja jest umieszczona w~pliku \verb|/etc/samba/smb.conf.default|. Plik
ten można skopiować do \verb|/etc/samba/smb.conf|, a~następnie edytować. Lista folderów
i~drukarek współdzielonych jest dostępna w~sekcji \verb|Share Definitions|.

O~ile system samba wykorzystuje system logowania Linux, wykorzystuje on oddzielną bazę haseł.
Hasła można zmienić korzystając z~\verb|smbpasswd|.

\end{document}


