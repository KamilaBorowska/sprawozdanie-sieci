% -*- latex -*-
% spr.tex

% Wymagane pola:
% - Imie Nazwisko
% - grupa (dzin+godzina zaj.)
% - Nazwa i numer cwiczenia)
% - data

% Wymagane jest umieszczenie źródeł sprawozdania (plik *.tex, źródła rysunków)
% w ktalogu ~/labsk. Pliki tekstowe muszą byc w formacie pl_PL.UTF-8
% (nie wiem jak plik tekstowy może być w formacie pl_PL, to jest locale,
%  chyba chodziło po prostu o UTF-8?)

% Wymagany jest wydruk z obu stron kartki!
\documentclass[a4paper,11pt]{article}

\usepackage{polski}
\usepackage{hyperref}
\usepackage[utf8]{inputenc}     % niepotrzebne gdy pierwsza linia pliku to "%& --translate-file=il2-pl"

\hoffset=-3.0cm                 % Mniejszy lewy margines
\textwidth=18cm                 % szerzej
\evensidemargin=0pt

\voffset=-3cm                   % Mniejszy gorny margines
\textheight=27cm                % szerzej wzdluz

\setlength{\parindent}{0pt}             % No paragraph indentation
\setlength{\parskip}{\medskipamount}    % Space between paragraphs
\raggedbottom                           % bez rozciagania strony

\title{Sprawozdanie z sieci -- ćwiczenie 5\\Poniedziałek, 10:00}
\author{Konrad Borowski\\Paweł Szczepanowski}

\begin{document}
\thispagestyle{empty}                   % bez numeracji stron

\maketitle

\section{Wstęp}

Celem zadania było podłączenie serwisów plikowych i~stworzenie serwerów
plików SMB, NFS i~iSCSI.

\subsection{Podłączanie zasobów NFS w~systemie Arch Linux}

Najpierw trzeba zainstalować program obsługujący system plików NFS.

\begin{verbatim}
pacman -S nfs-utils
\end{verbatim}

Można podłączyć zasób sieciowy za pomocą polecenia \verb|mount|.
Lokalizacja folderu sieciowego jest podzielona na dwie części,
pierwsza to nazwa lokalizacji sieciowej, a druga to ścieżka.
Opcja \verb|nolock| wyłącza mechanizm blokowania plików który jest
problematyczny w~warunkach sieciowych.

\begin{verbatim}
root@arch-nfs / # mount -o nolock vol:/ /mnt

vol:/ on /mnt type nfs (rw,relatime,vers=3,rsize=65536,
wsize=65536,namlen=255,hard,nolock,proto=tcp,timeo=600,
retrans=2,sec=sys,mountaddr=10.146.146.3,mountvers=3,
mountport=790,mountproto=udp,local_lock=all,addr=10.146.146.3)
\end{verbatim}

Następnie można zobaczyć listę zamontowanych folderów
używając polecenia \verb|df -h|.

\begin{verbatim}
root@arch-nfs / # df -h
System plików            rozm. użyte dost. %uż. zamont. na
[...]
vol:/                     125G  114G  1,3G  99% /mnt
\end{verbatim}

\section{Serwer NFS w~systemie Arch Linux}

Lista plików eksportowanych jest przechowywana w~pliku \verb|/etc/exports|.
Dla przykładu jeżeli chcemy mieć folder \verb|/srv/nfs| dostępny dla lokalnej
sieci 10.0.0.0 można zapisać następującą treść w~pliku \verb|/etc/exports|.

\begin{verbatim}
/srv/nfs       10.0.0.0/8(rw,fsid=root,no_subtree_check)
\end{verbatim}

A następnie uruchomić serwer.

\begin{verbatim}
sudo service nfs-server start
\end{verbatim}

Można sprawdzić widoczność używając następującego polecenia.

\begin{verbatim}
showmount -e localhost
\end{verbatim}

\section{Serwer SMB}

Można zainstalować go korzystając z~polecenia.

\begin{verbatim}
sudo pacman -S samba
\end{verbatim}

Jego domyślna konfiguracja jest umieszczona w~pliku \verb|/etc/samba/smb.conf.default|. Plik
ten można skopiować do \verb|/etc/samba/smb.conf|, a~następnie edytować. Lista folderów
i~drukarek współdzielonych jest dostępna w~sekcji \verb|Share Definitions|.

O~ile system samba wykorzystuje system logowania Linux, wykorzystuje on oddzielną bazę haseł.
Hasła można zmienić korzystając z~\verb|smbpasswd|.

\end{document}
