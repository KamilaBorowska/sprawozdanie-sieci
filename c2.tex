% -*- latex -*-
% spr.tex

% Wymagane pola:
% - Imie Nazwisko
% - grupa (dzin+godzina zaj.)
% - Nazwa i numer cwiczenia)
% - data

% Wymagane jest umieszczenie źródeł sprawozdania (plik *.tex, źródła rysunków)
% w ktalogu ~/labsk. Pliki tekstowe muszą byc w formacie pl_PL.UTF-8
% (nie wiem jak plik tekstowy może być w formacie pl_PL, to jest locale,
%  chyba chodziło po prostu o UTF-8?)

% Wymagany jest wydruk z obu stron kartki!
\documentclass[a4paper,11pt]{article}

\usepackage{polski}
\usepackage[utf8]{inputenc}     % niepotrzebne gdy pierwsza linia pliku to "%& --translate-file=il2-pl"

\hoffset=-3.0cm                 % Mniejszy lewy margines
\textwidth=18cm                 % szerzej
\evensidemargin=0pt

\voffset=-3cm                   % Mniejszy gorny margines
\textheight=27cm                % szerzej wzdluz

\setlength{\parindent}{0pt}             % No paragraph indentation
\setlength{\parskip}{\medskipamount}    % Space between paragraphs
\raggedbottom                           % bez rozciagania strony

\title{Sprawozdanie z sieci -- ćwiczenie 2\\Poniedziałek, 10:00}
\author{Konrad Borowski\\Paweł Szczepanowski}

\begin{document}
\thispagestyle{empty}                   % bez numeracji stron

\maketitle

\section{Wstęp}

Tworzenie sieci logicznej na podstawie logicznych aliasów można
przeprowadzić za pomocą polecenia ifconfig. Pozwala to na połączenie
dwóch interfejsów inną drogą.

\section{Tworzenie sieci po stronie grml}

\begin{verbatim}
ifconfig enp0s3:1 10.20.30.15/24
\end{verbatim}

\section{Tworzenie sieci po stronie FreeBSD}

\begin{verbatim}
ifconfig em0 10.20.30.16 alias netmask 255.255.255.0
\end{verbatim}

Polecenia te tworzą sieci z maską podsieci 255.255.255.0, ustalaną przez
argument \verb|netmask| na FreeBSD i~zakres sieci \verb|/24| w GRML.

Następnie można sprawdzić działanie sieci.

\section{Sprawdzenie działania po stronie grml}

\begin{verbatim}
root@malparkage ~ # ifconfig
...

enp0s3:1: flags=4163<UP,BROADCAST,RUNNING,MULTICAST>  mtu 1500
        inet 10.20.30.15  netmask 255.255.255.0  broadcast 10.20.30.255
        ether 08:00:27:56:c4:ef  txqueuelen 1000  (Ethernet)

...

root@malparkage ~ # ping -c 4 10.20.30.16
PING 10.20.30.16 (10.20.30.16) 56(84) bytes of data.
64 bytes from 10.20.30.16: icmp_seq=1 ttl=64 time=0.506 ms
64 bytes from 10.20.30.16: icmp_seq=2 ttl=64 time=0.460 ms
64 bytes from 10.20.30.16: icmp_seq=3 ttl=64 time=0.514 ms
64 bytes from 10.20.30.16: icmp_seq=4 ttl=64 time=0.587 ms

--- 10.20.30.16 ping statiroot@:~ # traceroute 10.20.30.15
traceroute to 10.20.30.15 (10.20.30.15), 64 hops max, 40 byte packets
 1  non-reg-LAN10 (10.20.30.15)  0.605 ms  0.486 ms  0.455 msstics ---
4 packets transmitted, 4 received, 0% packet loss, time 3010ms
rtt min/avg/max/mdev = 0.460/0.516/0.587/0.053 ms
\end{verbatim}

\section{Sprawdzenie działania po stronie FreeBSD}

\begin{verbatim}
root@:~ # ifconfig
em0: flags=8843<UP,BROADCAST,RUNNING,SIMPLEX,MULTICAST> metric 0 mtu 1500
	options=5059b<RXCSUM,TXCSUM,VLAN_MTU,VLAN_HWTAGGING,VLAN_HWCSUM,TSO4,LRO,VLAN_HWFILTER,VLAN_HWTSO>
	ether 08:00:27:e7:06:f8
	inet 10.146.50.139 netmask 0xffff0000 broadcast 10.146.255.255
	inet 10.20.30.16 netmask 0xffffff00 broadcast 10.20.30.255 
	nd6 options=29<PERFORMNUD,IFDISABLED,AUTO_LINKLOCAL>
	media: Ethernet autoselect (1000baseT <full-duplex>)
	status: active

root@:~ # ping -c 4 10.20.30.15
PING 10.20.30.15 (10.20.30.15): 56 data bytes
64 bytes from 10.20.30.15: icmp_seq=0 ttl=64 time=0.223 ms
64 bytes from 10.20.30.15: icmp_seq=1 ttl=64 time=0.585 ms
64 bytes from 10.20.30.15: icmp_seq=2 ttl=64 time=0.539 ms
64 bytes from 10.20.30.15: icmp_seq=3 ttl=64 time=0.547 ms

--- 10.20.30.15 ping statistics ---
4 packets transmitted, 4 packets received, 0.0% packet loss
round-trip min/avg/max/stddev = 0.223/0.474/0.585/0.146 ms

root@:~ # traceroute 10.20.30.15
traceroute to 10.20.30.15 (10.20.30.15), 64 hops max, 40 byte packets
 1  non-reg-LAN10 (10.20.30.15)  0.605 ms  0.486 ms  0.455 ms
\end{verbatim}

\end{document}
